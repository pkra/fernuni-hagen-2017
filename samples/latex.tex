\documentclass{article}
\usepackage{amsmath}
\usepackage{braket}
\usepackage[T1]{fontenc}
\usepackage[utf8]{inputenc}

\begin{document}

After a time $t$, the ground state $\ket{g}$ and the excited state $\ket{e}$ will each have accumulated a phase that is proportional to their energies:

\begin{align}
                \ket{\psi(0)} \to \ket{\psi(t)} = \frac{e^{-iE_1 t/\hbar}}{\sqrt{2}} \ket{g} + \frac{e^{-iE_2 t/\hbar}}{\sqrt{2}} \ket{e}\, .
\end{align}

We can take out the factor $e^{-iE_1 t/\hbar}$ as a global unobservable phase, and obtain

\begin{align}\label{eq:atomequator}
                \ket{\psi(t)} = \frac{1}{\sqrt{2}} \ket{g} + \frac{e^{-i(E_2 - E_1)t/\hbar}}{\sqrt{2}} \ket{e}\, .
\end{align}

\end{document}
